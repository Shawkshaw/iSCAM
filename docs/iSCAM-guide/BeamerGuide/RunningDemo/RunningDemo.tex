

%!TEX root = /Users/stevenmartell/Documents/CURRENT PROJECTS/iSCAM-trunk/docs/iSCAM-guide/overView.tex
\section{Running Examples} 

% (fold)
\label{sec:running_examples} 
\begin{frame}
	\frametitle{Running examples} Examples in \texttt{iSCAM-trunk/examples} 
	\begin{itemize}
		\item \texttt{Demo} 
		\item \texttt{Hake} 
	\end{itemize}
\end{frame}

\subsection{Demo Model} 

% (fold)
\label{sub:demo_model}
\begin{frame}
	\frametitle{Demo} 
	\begin{itemize}
		\item The \texttt{Demo} directory is not present in the examples when you first checkout a copy of \iscam\ from the svn repository. 
		\item To build the \texttt{Demo directory} cd to the examples directory and use \texttt{./makeproject Demo} 
	\end{itemize}
	\begin{figure}
		[htbp] \centering 
		\includegraphics[height=1.75in]{screenCaptures/TermDemo.pdf} \caption{Using \texttt{makeproject} command to create \texttt{Demo}.} \label{fig:screenCaptures_TermDemo} 
	\end{figure}
\end{frame}
\begin{frame}
	\frametitle{Running the ADMB model in Demo} 
	\begin{itemize}
		\item cd to the \texttt{examples/Demo/DATA} directory 
		\item type \texttt{make} at the command line 
	\end{itemize}
	\begin{figure}
		[htbp] \centering 
		\includegraphics[height=1.75in]{screenCaptures/Term-catage.pdf} \caption{Terminal output after the Demo model has run} \label{fig:screenCaptures_Term-catage} 
	\end{figure}
\end{frame}

% subsection demo_model (end)
\subsection{Makefile} 

% (fold)
\label{sub:makefile} 
\begin{frame}
	[shrink=10] \frametitle{More on using \texttt{Makefile} } A makefile is a Unix utility that automatically executes a set of shell commands (rules). \underline{Target} rules are executed based on \underline{dependencies}.
	\begin{block}
		{Targets} 
		\begin{itemize}
			\item all: copy executable and run model with DAT \& ARG 
			\item run: copy executable and force a run 
			\item mcmc: copy executable and run \texttt{mcmc} and \texttt{mceval} 
			\item retro: copy executable and run retrospective R-script 
			\item clean: remove executable \& other ADMB output files 
		\end{itemize}
	\end{block}
	\begin{block}
		{Dependencies} 
		\begin{itemize}
			\item EXEC - the name of the executable 
			\item CTL - the name of the control file 
		\end{itemize}
	\end{block}
	If the dependencies change then running make will execute the target scripts, otherwise there is no need to re-run the model. 
\end{frame}

\lstset{language=make} 
\begin{frame}
	[fragile] \frametitle{Setting up \texttt{Makefile}} User must supply variable Definitions in the Makefile: 
	\begin{lstlisting}
		[cap=Makefile Defs] EXEC = iscam prefix = ../../../dist DAT = RUN.dat CTL = ControlTemplate ARG = MCFLAG = -mcmc 10000 -mcsave 100 -nosdmcmc NR = 4 
	\end{lstlisting}
	\texttt{EXEC} is the program name, \texttt{prefix} is the (relative) path to the \texttt{dist} directory, \texttt{DAT} is the data file, \texttt{CTL} is the name of the control file, \texttt{ARG} optional command line argument (e.g., make run ARG="-nohess"), \texttt{MCFLAG} is the arguments for \texttt{make mcmc}, and \texttt{NR} is number of retrospective years (e.g., \texttt{make retro}). 
\end{frame}

\defverbatim[colored]
\makesmart{

%
\begin{lstlisting}
	[frame=single,emph={ga},emphstyle=\color{olive}] bash-3.2$ make make: Nothing to be done for `all'. bash-3.2$ 
\end{lstlisting}
}

%
\defverbatim[colored]
\makearg{

%
\begin{lstlisting}
	[frame=single,emph={ga},emphstyle=\color{olive}] bash-3.2$ make run ARG = "-est -nox" ... ******************************************* --Start time: Tue Nov 29 11:51:10 2011
	
	--Finish time: Tue Nov 29 11:51:11 2011
	
	--Runtime: 0 hours, 0 minutes, 1 seconds --Number of function evaluations: 424 --Results are saved with the base name: ControlTemplate ******************************************* bash-3.2$ 
\end{lstlisting}
}

%
\begin{frame}
	[fragile] \frametitle{Using \texttt{make} at the command line} \only<1>{ Makefiles are smart, will only execute rules if the dependencies change: 
	\vfill 
	\makesmart }
	
	%%
	\only<2>{ You can change the Makefile Defs at the command line: 
	\vfill 
	\makearg } 
\end{frame}
\begin{frame}
	[fragile,shrink=30] \frametitle{Parallel execution with \texttt{Make}} Run multiple models in SUBDIR using: \texttt{make -j4}\\
	The "-j#" option specifies the number of processors to use.\\
	\texttt{SUBDIR} is the list of subdirectories in \texttt{DATA} (one for each model)\\
	\begin{lstlisting}
		[frame=single] ## Makefile for running models ## Author: steven martell <martell.steve@gmail.com>
		
		## Macros SUBDIR = CC PRD QCI SOG WCVI AREA27 AREA2W TARGET = .PHONY: default $(SUBDIR) mcmc
		
		## Targets default: $(SUBDIR) $(SUBDIR): cd $@ && $(MAKE) $(TARGET)
		
		.PHONY: clean clean_files := $(foreach dir,$(SUBDIR),$(dir)/clean)
		
		clean: $(clean_files) $(clean_files): cd $(@D) && $(MAKE) clean 
	\end{lstlisting}
	Sorry does not work on WINDOZE! 
\end{frame}

% subsection makefile (end)
\begin{frame}
	\frametitle{Using \texttt{guiView}} In the R directory, source the iSCAM.R file in R $>$guiView() 
	\begin{figure}
		[htbp] \centering 
		\includegraphics[height=2.5in]{screenCaptures/guiView.pdf} \caption{R gui for \iscam} \label{fig:screenCaptures_guiView} 
	\end{figure}
\end{frame}

% section running_examples (end)
