%
%  overView
%
%  Created by Steven James Dean Martell on 2011-11-12.
%  Copyright (c) 2011 UBC Fisheries Centre. All rights reserved.
%
\documentclass[sans]{beamer}
%\usetheme{Madrid} % My favorite!
%\usetheme{PaloAlto}
\usetheme{Goettingen}
%\usetheme{Hannover}
%\usetheme{Boadilla} % Pretty neat, soft color.
%\usetheme{default}
%\usetheme{Warsaw}
%\usetheme{Bergen} % This template has nagivation on the left
%\usetheme{Frankfurt} % Similar to the default 
%with an extra region at the top.
%\usecolortheme{seahorse} % Simple and clean template
%\usecolortheme{beaver}
%\usetheme{Darmstadt} % not so good
% Uncomment the following line if you want %
% page numbers and using Warsaw theme%
% \setbeamertemplate{footline}[page number]
%\setbeamercovered{transparent}
\setbeamercovered{invisible}
% To remove the navigation symbols from 
% the bottom of slides%
\setbeamertemplate{navigation symbols}{} 
%
\usepackage{color}
\usepackage{graphicx}
\usepackage{listings}
%\usepackage{bm}         % For typesetting bold math (not \mathbold)
%\logo{\includegraphics[height=0.6cm]{yourlogo.eps}}
%

%Custom Color theme
\definecolor{bottomcolour}{rgb}{0.32,0.3,0.38}
\definecolor{middlecolour}{rgb}{0.08,0.08,0.16}
\setbeamerfont{title}{size=\Huge}
\setbeamercolor{structure}{fg=white}
\setbeamertemplate{frametitle}[default][center]

\setbeamercolor{normal text}{bg=black, fg=white}
\setbeamertemplate{background canvas}[vertical shading]
[bottom=bottomcolour, middle=middlecolour, top=black]

\setbeamertemplate{itemize item}{\lower3pt\hbox{\Large\textbullet}}
\setbeamerfont{frametitle}{size=\huge}
%end of color theme

%code block
\newtheorem{code}{Code}

%Table of contents at begining of each section
\AtBeginSection[]
{
   \begin{frame}
       \frametitle{Outline}
       \tableofcontents[currentsection]
   \end{frame}
}


%iscam logo
\usepackage{color}%

\newcommand{\iscam}{$i$SC$\forall$M}%
	% \raisebox{0.75ex}{$i$}%
	% \textcolor{red}{\raisebox{0.25ex}{S}}%
	% \textcolor{green}{\raisebox{0.00ex}{C}}%
	% \textcolor{blue}{\raisebox{-.25ex}{A}}%
	% \raisebox{-.50ex}{M}%
	% }%


\title[\iscam]{An Introduction to \iscam}
\author{Steven Martell}
\institute[UBC Fisheries Centre]
{
University of British Columbia \\
\medskip
{\emph{s.martell@fisheries.ubc.ca}}
}
\date{\today}
% \today will show current date. 
% Alternatively, you can specify a date.
%
\begin{document}

%
\begin{frame}
\titlepage
\end{frame}
%


%!TEX root = /Users/stevenmartell1/Documents/iSCAM-project/docs/iSCAM-guide/BeamerGuide/overView.tex
\section{Installation} % (fold)
\label{sec:installation}


\subsection[Source code]{Obtaining source code} % (fold)
\label{sub:obtaining_source_code}

% \begin{frame}[containsverbatim]
% 	\frametitle{Obtaining \iscam\ source code}
% 	Source code available at:\\
% 	\url{http://code.google.com/p/iscam-project/}
% 	\vfill
% 	Use subversion to checkout a copy
% 	\vfill
% 	\underline{On Mac \& Linux, in Terminal:}
% 	\tiny
% 	\begin{verbatim}
% 		svn checkout http://iscam-project.googlecode.com/svn/trunk/			iscam-project-read-only
% 	\end{verbatim}
% 	\normalsize
% 	\vfill
% 	\underline{On Windows use a subversion client}
% 	\url{http://tortoisesvn.net/}
% \end{frame}

\begin{frame}[containsverbatim]
	\frametitle{Obtaining \iscam\ source code}
	The source code is maintained at github:\\
	\url{https://github.com/smartell/iSCAM}
	
	\underline{Prerequisites}
	\begin{itemize}
		\item A C++ compiler (preferably gcc)
		\item AD Model Builder (version 11.0 or later)
		\item R (version 2.15 or later)
		\begin{itemize}
			\item PBSmodelling package (and dependencies)
			\item Hmisc package (and dependencies)
		\end{itemize}
	\end{itemize}

	% ## Prerequisites
	% 	* A C++ compiler (preferably gcc as this is the compiler that iSCAM is developed on).
	% 	* AD Model Builder (version 11.0 or later)
	% 	* R (version 2.15 or later)
	% 		* PBSmodelling package (and dependencies)
	% 		* Hmisc package (and dependencies)
	% 	
\end{frame}

% subsection obtaining_source_code (end)

% \subsection{SVN commands} % (fold)
% \label{sub:svn_commands}
% \begin{frame}
% 	\frametitle{Useful SVN commands}
% 	Usage: \texttt{svn command}
% 	\vfill
% 	\begin{scriptsize}
% 	\begin{tabular}{ll}
% 		Command & Description\\
% 		\hline
% 		\texttt{checkout} & Check out a working copy from a repository.\\
% 		\texttt{info}     & Display information about a local or remote item.\\
% 		\texttt{update}   & Bring changes from the repository into the working copy.\\
% 		\texttt{log}      & Show the log messages for a set of revision(s) and/or file(s).\\
% 		\texttt{revert}   & Restore pristine working copy file (undo most local edits).\\
% 		\texttt{commit}   & Send changes from your working copy to the repository.\\
% 		\texttt{diff}     & Display the differences between two revisions or paths.\\
% 		\texttt{help}     & Display help (usage \texttt{svn help <command>})\\
% 		\hline
% 	\end{tabular}
% 	\end{scriptsize}
% \end{frame}
% % subsection svn_commands (end)

\subsection{GIT} % (fold)
\label{sub:git}
\begin{frame}
	\frametitle{Working with Distributed Version Control (git)}
	
	Before using git, I would highly recommend spending some time 
	learning how to use git.  There are many online resources and
	most of them can be found at: \url{http://git-scm.com/documentation} \\
	\vfill
	A video tutorial: \url{http://www.youtube.com/watch?v=ZDR433b0HJY} \\
	\vfill
	Cheat sheet: \url{http://cheat.errtheblog.com/s/git/}
	\vfill
	Read the Readme file for more instructions.
\end{frame}

\begin{frame}[fragile]
	\frametitle{Initial checkout}
	Essentially want to make a clone of the repository on your computer.
	\vfill
	
	\begin{tiny}
	\begin{verbatim}
		git clone  git://github.com/smartell/iSCAM.git
	\end{verbatim}
	\end{tiny}
	\vfill
	The above command will make a copy of the repository, including the directory structure, on your computer. It will also copy all of the branches and branch history. You may not wish to do this.  If you wish to clone only a single branch (i.e. master), then the following should be done.
	
	\begin{tiny}
	\begin{verbatim}
		mkdir iSCAM-project
		cd iSCAM-project
		git init
		git remote add -t master -f origin git://github.com/smartell/iSCAM.git
		git checkout master
	\end{verbatim}
	\end{tiny}
\end{frame}

% subsection git (end)

\subsection{Directories} % (fold)
\label{sub:directories}
\begin{frame}
	\frametitle{Directory structure}
	\begin{columns}
	%
	\begin{column}{2in}
	\texttt{
	\begin{itemize}
		\item iSCAM-project
		\begin{itemize}
			\item dist
			\only<2>{
			\begin{itemize}
				\item debug
				\item R
				\item release
			\end{itemize}
			}
			\item docs
			\only<3>{
			\begin{itemize}
				\item API
				\item iSCAM-guide
				\item userGuide
			\end{itemize}
			}
			\item examples
			\only<4>{
			\begin{itemize}
				\item 4VWXHerring
				\item Cusk
				\item ...
				\item Makefile
				\item makeproject
			\end{itemize}
			}
			\item fba
			\only<5>{
			\begin{itemize}
				\item BC-herring-2011
				\item makeproject
				\item ReadMe.txt
			\end{itemize}
			}
			\item scripts
			\only<6>{
			\begin{itemize}
				\item scripts
			\end{itemize}
			}
			\item src
			\only<7>{
			\begin{itemize}
				\item admb-code
				\item r-code
			\end{itemize}
			}
		\end{itemize}
	\end{itemize}
	}
	\end{column}
	
	\begin{column}{2in}
		\only<2>{\texttt{dist} contains the compiled ADMB code
		 in debug and release versions, and the R scripts for 
		 dealing with output.}
		
		\only<3>{\texttt{docs} contains directories for the users
		guide, this presentation, and the API documentation for 
		the source code.\\[1ex]
		
		The users guide and presentation is written in latex, and 
		the API is built using doxygen.}
		
		\only<4>{\texttt{examples} directory contains several different
		examples and a Makefile for running the examples. \\[1ex]
		\texttt{makeproject}
		is a Unix script for setting up a new example directory.}
		
		\only<5>{\texttt{fba} is a directory for ``full blown assessment''\\[1ex]
		The ReadMe.txt file documents the various projects, and  \texttt{makeproject}
		is a Unix script for setting up a new assessment directory. }
		
		\only<6>{\texttt{scripts} contains various scripts that are copied into
		assessment directories. }
		
		\only<7>{\texttt{src} contains directories for the ADMB source code and the R-code and source files for the R-package.}
	\end{column}
	%
	\end{columns}
\end{frame}
% subsection directories (end)

\subsection{Text Editors} % (fold)
\label{sub:text_editors}


\begin{frame}
	\frametitle{Editors}
	\only<1>{
	\underline{Windows}
	\begin{itemize}
		\item Textpad \url{http://www.textpad.com/}
	\end{itemize}
	\underline{Mac OSX}
	\begin{itemize}
		\item Textmate \url{http://macromates.com/}
	\end{itemize}
	\underline{Linux}
	\begin{itemize}
		\item Vim \url{http://www.vim.org/}
	\end{itemize}
	\underline{Cross platform}
	\begin{itemize}
		\item Emacs \url{http://www.gnu.org/s/emacs/}
		\item eclipse \url{http://www.eclipse.org/}
		\item sublime \url{http://www.sublimetext.com/}
	\end{itemize}
	}
	\only<2>{
	\begin{figure}[htbp]
		\centering
			\includegraphics[height=2.5in]{screenCaptures/TextMate.pdf}
		\caption{Textmate on Mac OSX}
		\label{fig:screenCaptures_TextMate}
	\end{figure}
	}
\end{frame}
% subsection text_editors (end)

% section installation (end)

%!TEX root = /Users/stevenmartell1/Documents/iSCAM-project/docs/iSCAM-guide/BeamerGuide/overView.tex


\section[Compiling]{Compiling Source Code} % (fold)
\label{sec:compiling_source_code}

\begin{frame}
	\frametitle{Compiling ADMB source code}
	What you need:
	\begin{itemize}
		\item C++ compiler (gcc recommended)
		\begin{itemize}
			\item Mac OSX: install Xcode from appstore
			\item Linux: \url{http://gcc.gnu.org/}
			\item Windoze: \url{http://www.mingw.org/}
		\end{itemize}
		\item ADMB libraries: \url{http://admb-project.org/downloads}
	\end{itemize}
	\vfill
	ADMB source code for \iscam\ found in:\\
	\texttt{./iSCAM-trunk/src/admb-code/}
\end{frame}

\begin{frame}
	\frametitle{Compiling from the command line}
	At the command line:
	\begin{itemize}[<+->]
		\item use cd to navigate to the ./iSCAM-trunk directory
		\item \underline{Linux or Mac OSX:} type Make
		\item \underline{Windows:} see \url{http://gnuwin32.sourceforge.net/packages/make.htm}
	\end{itemize}
	
	\only<1>{
	\vspace{-.25in}
	\begin{figure}[htbp]
		\centering
			\includegraphics[height=2.5in]{screenCaptures/Term_iSCAM-trunk.pdf}
		\caption{}
		\label{fig:screenCaptures_Term_iSCAM-trunk}
	\end{figure}
	}
	\only<2>{
	\vspace{-.25in}
	\begin{figure}[htbp]
		\centering
			\includegraphics[height=2.5in]{screenCaptures/Term_make.pdf}
		\caption{}
		\label{fig:screenCaptures_Term_make}
	\end{figure}
	}
	\only<3>{
	\vspace{0.25in}
	Using the make file will compile the \iscam\ source code and place copies of the code in the distribution directory ("dist")
	}
\end{frame}

\begin{frame}
	\frametitle{Using make on windoze machines}
	If you want to run makefiles on Windows that were written for Mac or Linux, you need to reinstall mingw.  Make sure to check off "Developer tools" and "C++ libraries" and "Objective C libraries".  Then run the mingw shell from the start menu and once inside that you can just type "make" as usual.
\end{frame}

\begin{frame}[fragile]
	\frametitle{Windoze, c/o Gerry Black}
	Download Cygwin from \url{http://cygwin.com/setup.exe}  (don't use the 1st site listed)

	During the install do the default install, except also include the developer folder.

	Run the cygwin cmd prompt, and then do the export below…

	OR Adam Cook found that using doing the following in MinGW also works.

	\begin{verbatim}
	 $export ADMB_HOME="C:\Program Files\ADMB-11"
	 
	 $cd ~/iscam-project/src/admb-code
	 
	 $make
	 
	 $cd ~/iscam-project/examples/ECODETECTIVE/DATA
	 
	 $make
	\end{verbatim}	
\end{frame}

% section compiling_source_code (end)




%!TEX root = /Users/stevenmartell/Documents/CURRENT PROJECTS/iSCAM-trunk/docs/iSCAM-guide/overView.tex
\section{Running Examples} 

% (fold)
\label{sec:running_examples} 
\begin{frame}
	\frametitle{Running examples} Examples in \texttt{iSCAM-trunk/examples} 
	\begin{itemize}
		\item \texttt{Demo} 
		\item \texttt{Hake} 
	\end{itemize}
\end{frame}

\subsection{Demo Model} 

% (fold)
\label{sub:demo_model}
\begin{frame}
	\frametitle{Demo} 
	\begin{itemize}
		\item The \texttt{Demo} directory is not present in the examples when you first checkout a copy of \iscam\ from the svn repository. 
		\item To build the \texttt{Demo directory} cd to the examples directory and use \texttt{./makeproject Demo} 
	\end{itemize}
	\begin{figure}
		[htbp] \centering 
		\includegraphics[height=1.75in]{screenCaptures/TermDemo.pdf} \caption{Using \texttt{makeproject} command to create \texttt{Demo}.} \label{fig:screenCaptures_TermDemo} 
	\end{figure}
\end{frame}
\begin{frame}
	\frametitle{Running the ADMB model in Demo} 
	\begin{itemize}
		\item cd to the \texttt{examples/Demo/DATA} directory 
		\item type \texttt{make} at the command line 
	\end{itemize}
	\begin{figure}
		[htbp] \centering 
		\includegraphics[height=1.75in]{screenCaptures/Term-catage.pdf} \caption{Terminal output after the Demo model has run} \label{fig:screenCaptures_Term-catage} 
	\end{figure}
\end{frame}

% subsection demo_model (end)
\subsection{Makefile} 

% (fold)
\label{sub:makefile} 
\begin{frame}
	[shrink=10] \frametitle{More on using \texttt{Makefile} } A makefile is a Unix utility that automatically executes a set of shell commands (rules). \underline{Target} rules are executed based on \underline{dependencies}.
	\begin{block}
		{Targets} 
		\begin{itemize}
			\item all: copy executable and run model with DAT \& ARG 
			\item run: copy executable and force a run 
			\item mcmc: copy executable and run \texttt{mcmc} and \texttt{mceval} 
			\item retro: copy executable and run retrospective R-script 
			\item clean: remove executable \& other ADMB output files 
		\end{itemize}
	\end{block}
	\begin{block}
		{Dependencies} 
		\begin{itemize}
			\item EXEC - the name of the executable 
			\item CTL - the name of the control file 
		\end{itemize}
	\end{block}
	If the dependencies change then running make will execute the target scripts, otherwise there is no need to re-run the model. 
\end{frame}

\lstset{language=make} 
\begin{frame}
	[fragile] \frametitle{Setting up \texttt{Makefile}} User must supply variable Definitions in the Makefile: 
	\begin{lstlisting}
		[cap=Makefile Defs] EXEC = iscam prefix = ../../../dist DAT = RUN.dat CTL = ControlTemplate ARG = MCFLAG = -mcmc 10000 -mcsave 100 -nosdmcmc NR = 4 
	\end{lstlisting}
	\texttt{EXEC} is the program name, \texttt{prefix} is the (relative) path to the \texttt{dist} directory, \texttt{DAT} is the data file, \texttt{CTL} is the name of the control file, \texttt{ARG} optional command line argument (e.g., make run ARG="-nohess"), \texttt{MCFLAG} is the arguments for \texttt{make mcmc}, and \texttt{NR} is number of retrospective years (e.g., \texttt{make retro}). 
\end{frame}

\defverbatim[colored]
\makesmart{

%
\begin{lstlisting}
	[frame=single,emph={ga},emphstyle=\color{olive}] bash-3.2$ make make: Nothing to be done for `all'. bash-3.2$ 
\end{lstlisting}
}

%
\defverbatim[colored]
\makearg{

%
\begin{lstlisting}
	[frame=single,emph={ga},emphstyle=\color{olive}] bash-3.2$ make run ARG = "-est -nox" ... ******************************************* --Start time: Tue Nov 29 11:51:10 2011
	
	--Finish time: Tue Nov 29 11:51:11 2011
	
	--Runtime: 0 hours, 0 minutes, 1 seconds --Number of function evaluations: 424 --Results are saved with the base name: ControlTemplate ******************************************* bash-3.2$ 
\end{lstlisting}
}

%
\begin{frame}
	[fragile] \frametitle{Using \texttt{make} at the command line} \only<1>{ Makefiles are smart, will only execute rules if the dependencies change: 
	\vfill 
	\makesmart }
	
	%%
	\only<2>{ You can change the Makefile Defs at the command line: 
	\vfill 
	\makearg } 
\end{frame}
\begin{frame}
	[fragile,shrink=30] \frametitle{Parallel execution with \texttt{Make}} Run multiple models in SUBDIR using: \texttt{make -j4}\\
	The "-j#" option specifies the number of processors to use.\\
	\texttt{SUBDIR} is the list of subdirectories in \texttt{DATA} (one for each model)\\
	\begin{lstlisting}
		[frame=single] ## Makefile for running models ## Author: steven martell <martell.steve@gmail.com>
		
		## Macros SUBDIR = CC PRD QCI SOG WCVI AREA27 AREA2W TARGET = .PHONY: default $(SUBDIR) mcmc
		
		## Targets default: $(SUBDIR) $(SUBDIR): cd $@ && $(MAKE) $(TARGET)
		
		.PHONY: clean clean_files := $(foreach dir,$(SUBDIR),$(dir)/clean)
		
		clean: $(clean_files) $(clean_files): cd $(@D) && $(MAKE) clean 
	\end{lstlisting}
	Sorry does not work on WINDOZE! 
\end{frame}

% subsection makefile (end)
\begin{frame}
	\frametitle{Using \texttt{guiView}} In the R directory, source the iSCAM.R file in R $>$guiView() 
	\begin{figure}
		[htbp] \centering 
		\includegraphics[height=2.5in]{screenCaptures/guiView.pdf} \caption{R gui for \iscam} \label{fig:screenCaptures_guiView} 
	\end{figure}
\end{frame}

% section running_examples (end)

% 
% 
% 
% \begin{frame}
% \frametitle{Motivation}
% \begin{block}
% {Why Beamer?}
% Does anybody need an introduction to Beamer? I don't think so.
% \end{block}
% \end{frame}
% %
% \begin{frame}
% \frametitle{Example of a Theorem}
% \begin{theorem}
% The quick brown fox jumps over the lazy dog.
% \end{theorem}
% \end{frame}
% %
% \begin{frame}[fragile] % Notice the [fragile] option %
% \frametitle{Verbatim}
% \begin{example}[Putting Verbatim]
% \begin{verbatim}
% \begin{frame}
% \frametitle{Outline}
% \begin{block}
% {Why Beamer?}
% Does anybody need an introduction to Beamer?
% I don't think so.
% \end{block}
% % Extra carriage return causes problem with verbatim %
% \end{frame}\end{verbatim} 
% \end{example}
% \end{frame}
%  
% \begin{frame}[fragile]  % notice the fragile option, since the body
% 			% contains a verbatim command
% Example of the \verb|\cite| command to give a reference is below:
% Example of citation using \cite{key1} follows on.
% \end{frame}
%  
% \begin{frame}
% \frametitle{References}
% \footnotesize{
% \begin{thebibliography}{99}
%  \bibitem[Label1, 2010]{key1} Author's name (1987)
%  \newblock Title of the paper.
%  \newblock \emph{Journal Name} 55(4), 765 -- 799.
% \end{thebibliography}
% }
% \end{frame}
%  
% \begin{frame}
% \centerline{The End}
% \end{frame}
% End of slides
\end{document}

