%!TEX root = /Users/stevenmartell/Documents/CURRENT PROJECTS/iSCAM-trunk/fba/BC-herring-2011/WRITEUP/BCHerring2011.tex


%\subsection*{Abstract}
%\addcontentsline{toc}{subsection}{Abstract}
%
%June 15, 2011.  Structure of this paper has changed a bit. This document will now consist of an assessment and forecast of the five major stocks and the two minor stocks.  There will be at least 5 appendixes that 1) describe the input data and the control files used for the assessment model, 2) a detailed description of \iscam, 3) a description of the methods used to develop the prior distribution, 4) simulation testing of the \iscam model, 5) moving toward the sustainable fisheries framework (see Cleary and Cox paper) and include discussion of the issues of developing an MSY-based framework for a multigear fishery with changing selectivities and natural mortality rates, and finally 6) a list of research recommendations.
%
%Summary:  Three major themes of the paper: 1) a comparing HCAM and iSCAM (where iSCAM is set up with nearly the same assumptions as 2010 HCAM assessment), 2) an iSCAM assessment with several scenarios addressing (a) q with various priors, (b) time-varying versus constant M, (c) alternative selectivity models, and (d) the interactions of all three of these confounded variables, and 3) an iSCAM assessment with the test fishery and seine roe fishery data separated into specific fleets.  The side by side comparison will examine similarities/differences between trends in biomass, fishing mortality rates, and residual fits to the spawn survey data an age-composition data.  These two models have some fundamental differences in the statistical assumptions about the catch-at-age data, so results are likely to be slightly different.  Results for all three themes will focus on reconstructing table 5 from last years assessment, with the addition of LRP and USRP to be compared with the cuttoffs and catch advice for low med and high recruitment.
%


\section*{Abstract}\addcontentsline{toc}{section}{Abstract}

Estimates of herring abundance in British Columbia (B.C.) waters  has been based on catch-age data and spawn survey abundance information.  These data are typically interpolated using a statistical catch-age framework; however, virtual methods (e.g., VPA) have been used in the past. This assessment also uses a statistical catch age model.  This document is broken into two parts: Part I deals with moving the herring assessments towards Canada's sustainable fisheries framework and introduces a new integrated statistical catch-age model for jointly estimating the abundance of Pacific herring and associated reference points to be used in the sustainable fisheries framework.  Part II of this document  implements this new assessment framework using the data for the five major and two minor regions.  Finally, we present catch advice based on decision tables that utilize poor, average, and good age-3 recruitment forecasts.

In Part I of this document we provide a very brief description of the new assessment framework (a full technical description of the model is provided in the Appendix of this document).  We then conduct some simulation testing with perfect information to  demonstrate that the model is capable of estimating all the parameters. We further explore precision and bias in parameter estimates based on simulation data with both observation and process errors.  We then parameterize the new assessment model such that the assumptions of the previous assessment model (Herring Catch Age Model, or HCAM) are mostly met and compare parameter estimates and estimates of spawning stock biomass (using data from 1951:2010).  Using data from the Strait of Georgia only, we then compare alternative assumptions about the spawn survey scaling coefficient ($q$), natural mortality and selectivity, and examine how these alternative assumptions influence estimates of key parameters (unfished spawning biomass, steepness, average natural mortality).  Relaxing assumptions about $q$ and natural mortality rates had the largest impacts on estimated parameters.  Lastly, we compared estimates of spawning stock biomass from HCAM with the new model for all five major areas to understand the subtle differences between the assumptions in the two models.% \footnote{At the time of writing this document and submitting it for peer review, an error was notices in the formulation of the weight-based selectivity function for the gill net fishery.  This error has been corrected for all of the results presented in Part II, but has still to be corrected for the model comparisons in Part I of this document.  Time permitting, these comparisons will be conducted prior to the assessment meeting, and if not at least presented at the September 7, 2011 meeting.}.

In Part II of this document we present updated data from the herring fisheries and surveys in 2011, a brief description of the analytical methods used to construct the decision tables, and present the results of the application of the new assessment model to the 2011 data. New this year is a Bayesian prior for the dive survey spawn index ($q$) and the development of this prior is detailed in the appendix.  The expected value of $q$ was estimated to be 0.587 with a standard deviation of 0.155. To summarize the overall fit to the model, maximum likelihood estimates derived quantities and residuals between observed and predicted variables are used.  Retrospective analysis (i.e., the sequential removal of the most recent data) is used as a diagnostic for model misspecification.   Catch advice (decision tables) are based on the median values of random samples from the joint posterior distribution and not the maximum likelihood estimates.  Visual inspection of the trace plots from the posterior samples and pair plots were used to judge if the samples were taken from a stationary distribution. Historically catch advice was based on cuttoff values that were derived from 1996 estimates of the unfished biomass (\bo, cuttoff values are set at 0.25\bo).  This assessment provides updated estimates of \bo\ and presents catch advice based on new cuttoff values.  An alternative decision table, where catch advice is based on old cuttoffs, is also presented.  

Median estimates of the 2011 spawning stock biomass is as follows: Haida Gwaii (HG) --16,579 t, Prince Rupert District (PRD) -- 27,046 t, Central Coast (CC) -- 14,666 t, Strait of Georgia (SOG) 125,261 t, West Coast Vancouver Island (WCVI) 14,679 t.  Implementation of the current harvest control rule (HCR) advises no fishing in HG under poor recruitment and no fishing in CC under poor and average recruitment.  Based on the 20\% harvest rate and application of the harvest control rule the estimated maximum available harvest ranges from 4,296 t in HG to 27,690t in SOG (assuming good recruitment).  Catch advice for the minor areas is based on a 10\% fixed exploitation rate with no cuttoffs and ranges from 91 t in Area 27 (assuming poor recruitment) to 614 t in Area 2W (assuming good recruitment).

%\section*{Executive summary}\addcontentsline{toc}{section}{Executive summary}
