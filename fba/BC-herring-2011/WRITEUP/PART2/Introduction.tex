cutoff%!TEX root = /Users/stevenmartell/Documents/CURRENT PROJECTS/iSCAM-trunk/fba/BC-herring-2011/WRITEUP/BCHerring2011.tex


\section{Introduction}

The objectives of this section of the report are: (1) present the data used in the 2011 assessment, (2) provide a summary overview of the integrated statistical catch-age model (hereafter, \iscam), (3) present the 2011 stock assessment and forecast for 2012, and (4) describe in detail the decision table used to provide advice to fisheries management.

BC herring are currently managed as five major stocks and 2 minor stocks (Figure \ref{Fig1}).  Annual catch advice for each of these areas is based on current estimates of stock status, and a 20\% exploitation rate if the post-fishery stock is above the cutoff level for the five major stocks and a 10\% exploitation rate for the two minor stocks.  Cutoff levels for the five major stocks historically were based on the 1996 estimate of  0.25$B_o$.  These cutoffs are currently thought to be more conservative 	than the suggested default Limit Reference Point of 0.4\bmsy\ \citep{dfo2006}. For example, \bmsy\ is normally in the range of 35\% of the unfished biomass for many fish stocks; therefore,  40\% of \bmsy is roughly 14\% of  unfished which is significantly lower than the 25\%$B_o$ that is currently used for Pacific herring.   Alternative cutoffs based on updated estimates of $B_o$ are also provided in this document.

This years assessment is based on a new model, \iscam, where alternative assumptions about survey $q$, and the form of the error distribution for the age-composition data are the major differences in comparison to the 2010 assessment using HCAM.  In addition to the changes in likelihoods, we also present an alternative parametrization of the gillnet selectivity to determine if the residual variation in gillnet age-composition data are better explained by systematic changes in the empirical weight-at-age data or selectivity has been relatively constant and natural mortality rates have varied over time.


In this part of the document, we first describe the five major and two minor Stock Assessment Regions (SARS) that comprise the BC herring stocks. We then present the input data used in this years assessment, briefly describe the analytical methods and diagnostics, describe the recruitment and catch forecasts, and the Harvest Control Rule (HCR) used for generating catch advice. We then present the maximum likelihood estimates of residual patterns and overall fits to the observations, summarize MSY based reference points and maximum likelihood estimates of \bo. Lastly, we present the results of integrating the joint posterior distribution, diagnostics for ensuring convergence, marginal parameter distributions (with prior distributions overlaid), and catch advice based on the median values of the joint posterior distribution.  The last section presents the data, MLE results, marginal distributions and catch advice for the two minor areas (the HCR in the minor area differs from the major areas).


\section{BC Herring Stocks}
The geographic boundaries used to delineate the B.C. herring stock assessment regions have remained consistent since 1993.  Boundaries and locations of the major stock and minor stock areas are identified in Figure \ref{Fig1}.  The Haida Gwaii (HG) or Queen Charlotte Islands (QCI2E) stock assessment region includes most of Statistical Area 2E, spanning from Cumshewa Inlet in the north to Louscoone Inlet in the south.  The Prince Rupert District (PRD) stock assessment region encompasses Statistical Areas 03 to 05.  The Central Coast (CC) assessment region separates the major migratory stocks from the minor spawning populations in the mainland inlets.  The Central Coast assessment region includes Statistical Area 07 plus Kitasu Bay in Area 06, Kwakshua Channel in Section 085 and Fitz Hugh Sound in Section 086.  The Strait of Georgia (SOG) stock assessment region includes all of Statistical Areas 14 to 19, 28, and 29 (excluding Section 293), Deepwater Bay and Okisollo Channel, both in Section 132, and Section 135.  The west coast of Vancouver Island (WCVI) assessment region encompasses Statistical Areas 23 to 25.  The minor stocks include all of Area 27 and Area 2W (excluding Louscoone Inlet in Section 006).  Current geographic stock boundaries are outlined in \cite{Midgley:2003fk}, although note that SOG sections 280 and 291 do not appear as they were added in 2006.

%%\begin{figure}[!tbp]
%%	% Requires \usepackage{graphicx}
%%	%\includegraphics[width=\textwidth]{Figs/HerringAreaMap.pdf}
%%	\includegraphics[width=0.95\textwidth]{../FIGS/PBSfigs/Assessment_Regions_2W_27_2010_HG.pdf}
%%	\caption{B.C. herring major stock areas: Haida Gwaii (HG or QCI 2E), Prince Rupert District (PRD), Central
%%Coast (CC), Strait of Georgia (SOG), West Coast Vancouver Island (WCVI), and minor stock areas: Area 2W and
%%Area 27.}\label{II:fig:map}
%%\end{figure}